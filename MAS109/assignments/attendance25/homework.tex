% ! TeX program = lualatex

\documentclass{homework}
% \usepackage{lua-visual-debug}
\usepackage{amsmath}
\usepackage{amssymb}
\usepackage{amsfonts}
\usepackage{enumitem}
\usepackage{mathtools}
\usepackage{ulem}
\usepackage{listings}
\usepackage{stackrel}
\usepackage{mathdots}

\usepackage{macros-common}

\allowdisplaybreaks

\title{Attendance Quiz \#25}
\subject{MAS109EF Introduction to Linear Algebra}
\studentid{20170058}
\name{Keonwoo Kim}
\date{\today}

\begin{document}
\maketitle

\newcommand{\row}{\operatorname{row}}
\newcommand{\col}{\operatorname{col}}
\newcommand{\rk}{\operatorname{rk}}
\newcommand{\nll}{\operatorname{null}}
\newcommand{\diag}{\operatorname{diag}}
\newcommand{\nullity}{\operatorname{nullity}}
\newcommand{\tr}{\operatorname{tr}}

\noindent\textit{Solution.}
\begin{align*}
    \lambda\mathbf v^T\mathbf u = \mathbf v^T (\lambda\mathbf u) = \mathbf v^T A\mathbf u = (A^T\mathbf v)^T\mathbf u = \mu\mathbf v^T\mathbf u
\end{align*}
so that $(\lambda-\mu)\mathbf v^T\mathbf u = 0$. Since $\lambda\ne \mu$, we have $\mathbf v^T\mathbf u = 0$, i.e., $\mathbf u\perp\mathbf v$.\hfill$\square$

\end{document}
