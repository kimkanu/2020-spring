% ! TeX program = lualatex

\documentclass{homework}
% \usepackage{lua-visual-debug}
\usepackage{amsmath}
\usepackage{amsfonts}
\usepackage{enumitem}
\usepackage{mathtools}
\usepackage{ulem}
\usepackage{stackrel}

\usepackage{macros-common}
\usepackage{macros-matrix}
\usepackage[bb]{macros-prob}

\title{Homework 1}
\subject{MAS583 Random Matrix Theory and Its Application}
\studentid{20170058}
\name{Keonwoo Kim}
\date{\today}

\begin{document}
\maketitle


\solution{
Prove that $\E[\angl{\mu, x^k}^2] - (\E\angl{\mu,x^k})^2 \to 0$ as $N\to\infty$.

\noindent\textcolor{white!60!black}{\rule{3cm}{0.2pt}}\hfill\\[5mm]
%
As we did in the calculation of $\E\angl{\mu, x^k}$, we can rephrase $\E[\angl{\mu, x^k}^2] - (\E\angl{\mu,x^k})^2$ as follows:
\begin{align*}
     & \E\bracket{\langle{\mu, x^k}\rangle^2} - \paren{\E\langle{\mu,x^k}\rangle}^2
    \\ &=
    \E\bracket[\Bigg]{\paren[\bigg]{\frac 1N \sum_{j=1}^N \lambda_j^k}^{\!\!2\,}} - \paren[\Bigg]{\frac 1 N\E\sum_{j=1}^N { \lambda_j^ k}}^{\!\!2}
    \\ &= \frac 1 {N^2} \paren{\E\bracket[\Bigg]{\paren[\bigg]{\sum_{j=1}^N \lambda_j^k}^{\!\!2\,}} - \paren[\Bigg]{\E\sum_{j=1}^N { \lambda_j^ k}}^{\!\!2}}
    \\ &= \frac 1 {N^2} \paren{\E\bracket{(\tr H^k)^2} - \paren[\Bigg]{\E\bracket{\tr H^k}}^{\!\!2}}
    \\ &= \frac 1 {N^2} \paren{\E\bracket[\Bigg]{\sum_{i_1,\dots,i_k=1}^N\sum_{i'_1,\dots,i'_k=1}^N H_{i_1,i_2}\cdots H_{i_k,i_1}H_{i'_1,i'_2}\cdots H_{i'_k,i'_1}} \right.
    \\ &\qquad\qquad \left.- \paren[\Bigg]{\E\sum_{i_1,\dots,i_k=1}^N H_{i_1,i_2}\cdots H_{i_k,i_1} }^{\!\!2}}
    \\ &= \frac 1 {N^2} {\sum_{\mathbf i,\mathbf i' \in \{1,2,\dots,N\}^k }\paren[\big]{ \E\bracket{T_{\mathbf i}T_{\mathbf i'} } - \E T_{\mathbf i}\,\E T_{\mathbf i'\!}}}
\end{align*}
where $T_{(i_1,\dots,i_k)} = H_{i_1,i_2}H_{i_2,i_3}\cdots H_{i_k,i_1}$.

Like a way we defined a graph for each $\mathbf i \in \{1,\dots,N\}^k$, we may associate a pair $(\mathbf i,\mathbf i')$ to a graph as follows:

\setlength{\leftskip}{2\parindent}
\noindent Let $\mathbf i = (i_1,\dots,i_k),\mathbf i' = (i'_1,\dots,i'_k) \in \{1,\dots,N\}^k$. Define the graph $G_{\mathbf i} = (V_{\mathbf i},E_{\mathbf i})$ associated with $\mathbf i$ where $V_{\mathbf i} = \{ i_j : j\in\{1,\dots,k\} \}$ and $E_{\mathbf i} =  \{ \{i_j,i_{j+1}\} : j\in \{1,\dots,k\} \}$ with $i_{k+1}\coloneqq i_1$. Also, define the graph $G_{\mathbf i,\mathbf i'} = (V_{\mathbf i, \mathbf i'}, E_{\mathbf i, \mathbf i'})$ associated with $\{\mathbf i,\mathbf i'\}$ where $V_{\mathbf i, \mathbf  i'} = V_{\mathbf i}\cup V_{\mathbf i'}$ and
$ E_{\mathbf i,\mathbf i'} = E_{\mathbf i} \cup E_{\mathbf i'}.$

Traversing the graph $G_{\mathbf i}$ or $G_{\mathbf i,\mathbf i'}$, let $N_{\mathbf i}$ or $N_{\mathbf i,\mathbf i'}(e)$ ($e\in E_{\mathbf i}$ or $e\in E_{\mathbf i,\mathbf i'}$) be the number of times the traverse passes $e$ (in any direction), respectively.

\setlength{\leftskip}{0cm}
With these definitions, we obtain
\begin{align*}
    \E\bracket{T_{\mathbf i}T_{\mathbf i'}} &= \prod_{\substack{e\in E_{\mathbf i,\mathbf i'} \\ \text{$e$ is a loop}}} \E\bracket{ H_e^{N_{\mathbf i,\mathbf i'} (e)} }\cdot \prod_{\substack{e\in E_{\mathbf i,\mathbf i'} \\ \text{$e$ is a non-loop}}} \E\bracket{ H_e^{N_{\mathbf i,\mathbf i'} (e)} }
    \\ &= \prod_{\substack{e\in E_{\mathbf i,\mathbf i'} \\ \text{$e$ is a loop}}} \E\bracket{ H_{11}^{N_{\mathbf i,\mathbf i'} (e)} }\cdot \prod_{\substack{e\in E_{\mathbf i,\mathbf i'} \\ \text{$e$ is a non-loop}}} \E\bracket{ H_{12}^{N_{\mathbf i,\mathbf i'} (e)} },
\end{align*}
where $H_e = H_{ij}$ if $e=\{i,j\}$, due to the identical distribution conditions. Similarly,
\begin{align*}
    \E\bracket{T_{\mathbf i}} &= \prod_{\substack{e\in E_{\mathbf i} \\ \text{$e$ is a loop}}} \E\bracket{ H_e^{N_{\mathbf i} (e)} }\cdot \prod_{\substack{e\in E_{\mathbf i} \\ \text{$e$ is a non-loop}}} \E\bracket{ H_e^{N_{\mathbf i} (e)} }
    \\ &= \prod_{\substack{e\in E_{\mathbf i} \\ \text{$e$ is a loop}}} \E\bracket{ H_{11}^{N_{\mathbf i} (e)} }\cdot \prod_{\substack{e\in E_{\mathbf i} \\ \text{$e$ is a non-loop}}} \E\bracket{ H_{12}^{N_{\mathbf i} (e)} }.
\end{align*}

Because $\E{H_{11}}=\E{H_{12}} = 0$, unless $N_{\mathbf i,\mathbf i'} (e)= N_{\mathbf i}(e) +N_{\mathbf i'}(e) \ge 2$ for all $e\in E_{\mathbf i,\mathbf i'}$, we have $\E\bracket{T_{\mathbf i}T_{\mathbf i'} } - \E T_{\mathbf i}\,\E T_{\mathbf i'\!} = 0$. Also when $E_{\mathbf i} \cap E_{\mathbf i'}= \emptyset$, we have $\E[T_{\mathbf i}T_{\mathbf i'}] = \E[T_{\mathbf i}] \E[T_{\mathbf i'}]$ due to the independence conditions. Moreover, when there is a bijection on $\{1,\dots, N\}$ which maps $\mathbf i$ to $\mathbf j$ and $\mathbf i'$ to $\mathbf j'$, then we have
$$ \E\bracket{T_{\mathbf i}T_{\mathbf i'} } - \E T_{\mathbf i}\,\E T_{\mathbf i'\!} = \E\bracket{T_{\mathbf j}T_{\mathbf j'} } - \E T_{\mathbf j}\,\E T_{\mathbf j'\!} $$
due to the identical distribution (by applying the bijection on the product). So, this defines an equivalence relation on $\paren{\set{1,\dots,N}^k}^2$.

Now, we will count those equivalence classes (of $(\mathbf i,\mathbf i')$'s) by $|V_{\mathbf i,\mathbf i'}| (\le 2k)$. Let us $\mathcal G_v$ denote the set of all representatives for equivalence classes of $a_{\mathbf i,\mathbf i'}$'s (defined by the bijection on $\{1,\dots,N\}$) with $| V_{\mathbf i,\mathbf i'} | = v$, $N_{\mathbf i,\mathbf i'}(e)\ge 2$ for every $e\in E_{\mathbf i,\mathbf i'}$, and $E_{\mathbf i}\cap E_{\mathbf i'} \ne \emptyset$. Note that the cardinality of an equivalence class is exactly $v! \binom N v$, if $N$ is sufficiently large, that is, $N\ge v$.

Using this observation, we have (when $N\ge 2k$)
\begin{align*}
     & \E\bracket{\langle{\mu, x^k}\rangle^2} - \paren{\E\langle{\mu,x^k}\rangle}^2
    \\ &= \frac 1 {N^2} {\sum_{\mathbf i,\mathbf i' \in \{1,2,\dots,N\}^k }\paren[\big]{ \E\bracket{T_{\mathbf i}T_{\mathbf i'} } - \E T_{\mathbf i}\,\E T_{\mathbf i'\!}}}
    \\ &= \frac 1 {N^2} {\sum_{\mathbf i,\mathbf i' \in \{1,2,\dots,N\}^k }\paren[\big]{ \E\bracket{T_{\mathbf i}T_{\mathbf i'} } - \E T_{\mathbf i}\,\E T_{\mathbf i'\!}}}
    \\ &= \frac 1 {N^2} {\sum_{v=1}^{2k} v!\binom N v \sum_{(\mathbf i,\mathbf i') \in \mathcal G_v} \paren[\big]{ \E\bracket{T_{\mathbf i}T_{\mathbf i'} } - \E T_{\mathbf i}\,\E T_{\mathbf i'\!}}}
    \\ &= \frac 1 {N^2} \sum_{v=1}^{2k} v!\binom N v \sum_{(\mathbf i,\mathbf i') \in \mathcal G_v} \biggl(
        \prod_{\substack{e\in E_{\mathbf i,\mathbf i'} \\ \text{$e$ is a loop}}}\mkern-10mu \E\bracket{ H_{11}^{N_{\mathbf i,\mathbf i'} (e)} }\cdot \mkern-30mu\prod_{\substack{e\in E_{\mathbf i,\mathbf i'} \\ \text{$e$ is a non-loop}}} \mkern-20mu\E\bracket{ H_{12}^{N_{\mathbf i,\mathbf i'} (e)} } \biggr.
        \\&\biggl. - 
        \prod_{\substack{e\in E_{\mathbf i} \\ \text{$e$ is a loop}}} \mkern-10mu\E\bracket{ H_{11}^{N_{\mathbf i} (e)} }\cdot \mkern-30mu\prod_{\substack{e\in E_{\mathbf i} \\ \text{$e$ is a non-loop}}} \mkern-20mu\E\bracket{ H_{12}^{N_{\mathbf i} (e)} }\cdot \mkern-20mu
        \prod_{\substack{e\in E_{\mathbf i'} \\ \text{$e$ is a loop}}}\mkern-10mu \E\bracket{ H_{11}^{N_{\mathbf i} (e)} }\cdot \mkern-30mu\prod_{\substack{e\in E_{\mathbf i'} \\ \text{$e$ is a non-loop}}} \mkern-20mu\E\bracket{ H_{12}^{N_{\mathbf i} (e)} }
        \biggr)
    \\ &= \frac 1 {N^{k + 2}} \sum_{v=1}^{2k} v!\binom N v \sum_{(\mathbf i,\mathbf i') \in \mathcal G_v} \biggl(
        \prod_{\substack{e\in E_{\mathbf i,\mathbf i'} \\ \text{$e$ is a loop}}}\mkern-10mu \E\bracket{ \hat H_{11}^{N_{\mathbf i,\mathbf i'} (e)} }\cdot \mkern-30mu\prod_{\substack{e\in E_{\mathbf i,\mathbf i'} \\ \text{$e$ is a non-loop}}} \mkern-20mu\E\bracket{ \hat H_{12}^{N_{\mathbf i,\mathbf i'} (e)} } \biggr.
        \\&\biggl. - 
        \prod_{\substack{e\in E_{\mathbf i} \\ \text{$e$ is a loop}}} \mkern-10mu\E\bracket{ \hat H_{11}^{N_{\mathbf i} (e)} }\cdot \mkern-30mu\prod_{\substack{e\in E_{\mathbf i} \\ \text{$e$ is a non-loop}}} \mkern-20mu\E\bracket{ \hat H_{12}^{N_{\mathbf i} (e)} }\cdot \mkern-20mu
        \prod_{\substack{e\in E_{\mathbf i'} \\ \text{$e$ is a loop}}}\mkern-10mu \E\bracket{ \hat H_{11}^{N_{\mathbf i} (e)} }\cdot \mkern-30mu\prod_{\substack{e\in E_{\mathbf i'} \\ \text{$e$ is a non-loop}}} \mkern-20mu\E\bracket{ \hat H_{12}^{N_{\mathbf i} (e)} }
        \biggr)
\end{align*}
where $\hat H_{ij} \coloneqq N^{-1/2}H_{ij}\sim \mathcal N(0,1)$. Since any $k$-th moment of a standard normal random variable are finitely well-defined, $\sum_{(\mathbf i,\mathbf i')\in \mathcal G_v} (\cdots)$ in the last line of the equation above does not depend on $N$. Denoting those terms (independent of $N$) as $C_v$, we have
$$ \E\bracket{\langle{\mu, x^k}\rangle^2} - \paren{\E\langle{\mu,x^k}\rangle}^2 = \sum_{v=1}^{2k} C_v\cdot v!\cdot N^{-(k+2)}\binom N v. $$
Therefore, it suffices to prove that $\mathcal G_v = \emptyset$ so that $C_v = 0$ for $v\ge k+2$, since other (lower degree) terms disappears as $N\to \infty$, since $\binom N v \sim N^{v}$.

Suppose $(\mathbf i, \mathbf i') \in \mathcal G_v$. Since $E_{\mathbf i} \cap E_{\mathbf i'}\ne\emptyset$, $G_{\mathbf i,\mathbf i'}$ is connected, with $v$ vertices and $\le k$ edges, as every edge should be passed more than once during traverse. Since $v = |V(G_{\mathbf i,\mathbf i})| \le |E(G_{\mathbf i,\mathbf i})| + 1 \le k+1$ for a connected graph $G_{\mathbf i,\mathbf i}$, we have $\mathcal G_v = \emptyset$ when $v \ge k+2$. This completes the proof.\hfill $\square$
}


\end{document}
